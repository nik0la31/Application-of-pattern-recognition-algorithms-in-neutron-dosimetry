\documentclass[12pt,a4paper,serbian,oneside]{book}
\usepackage[utf8x]{inputenc}
\usepackage[T2A]{fontenc}
\usepackage{amsmath}
\usepackage{amssymb}
\usepackage{textcomp}
\usepackage{amsfonts}
\usepackage{graphicx}
\usepackage{ucs}
\usepackage{listings}
\usepackage[serbian]{babel}
\usepackage{pdfsync}
\usepackage[left=2.5cm,right=2.5cm,top=2.5cm,bottom=2.5cm]{geometry}


% Komanda za horizontal ruler
\newcommand{\HRule}{\rule{\linewidth}{0.5mm}}

%
% Definicija fiksiranih reci
%
\addto\captionsserbian{%
 \def\prefacename{Предговор}%
 \def\refname{Списак литературе}%
 \def\abstractname{Сажетак}%
 \def\bibname{Литература}%
 \def\chaptername{Глава}%
 \def\appendixname{Додатак}%
 \def\contentsname{Садржај}%
 \def\listfigurename{Списак слика}%
 \def\listtablename{Списак табела}%
 \def\indexname{Регистар}%
 \def\figurename{Слика}%
 \def\tablename{Табела}%
 \def\partname{Део}%
 \def\enclname{Прилози}%
 \def\ccname{Копије}%
 \def\headtoname{Прима}%
 \def\pagename{Страна}%
 \def\seename{Види}%
 \def\alsoname{Види такође}%
 \def\proofname{Доказ}%
 \def\glossaryname{Речник непознатих речи}
 \def\contentsname{Садржај}%
 }%

%
% Podesavanja paketa za listinge
%
\renewcommand\lstlistingname{Изворни код}
\lstset {
	basicstyle=\footnotesize\ttfamily,
	numbers=left,
	numberstyle=\tiny,
	%stepnumber=2,
	numbersep=5pt,
	tabsize=2,
	extendedchars=true,
	breaklines=true,
	keywordstyle=\color{blue},
	frame=b,
	stringstyle=\color{white}\ttfamily,
	showspaces=false,
	showtabs=false,
	xleftmargin=17pt,
	framexleftmargin=14pt,
	framexrightmargin=3pt,
	framexbottommargin=4pt,
	framextopmargin=0pt,
	%backgroundcolor=\color{lightgray},
	showstringspaces=false,
          commentstyle=\color{green!50!black},
}

\lstloadlanguages {
	% Check Documentation for further languages ...
	%[Visual]Basic
	%Pascal
	%C
	%C++
	Java,
	XML
	%HTML
}

%
% Podesavanje okvira listinga
%
\usepackage{color}
\usepackage{xcolor}
\usepackage{caption}
\DeclareCaptionFont{white}{\color{white}}
\DeclareCaptionFormat{listing}{\colorbox[cmyk]{0.43, 0.35, 0.35,0.01}{\parbox{\textwidth}{\hspace{12pt}#1#2#3}}}
\captionsetup[lstlisting]{format=listing,labelfont=white,textfont=white, singlelinecheck=false, margin=0pt, font={bf,footnotesize} }

%
% Dupla kosa crta
%
\usepackage{stmaryrd}

%
% Pocetak dokumenta
%
\begin{document}

\input{naslovna.tex}

\tableofcontents
\newpage

\chapter{Увод}

\chapter{Обрада слика употребом OpenCV библиотеке}

OpenCV (Open Source Computer Vision Library) је библиотека отвореног кода која садржи имплеметације више стотина алгоритама рачунарског вида (computer vision). Библиотека је написана у C++ програмском језику, али подржава и итерфејс ка C, Python, MATLAB програмским језицима, а у разоју су и интерфејси за CUDA и OpenCL језике. Подржана је на свим видећим операстивним системима Windows, Linux, Android и Mac OS. Библиотека је објављена под BSD лиценцом, те је стога погодна  и за академску и за комерцијалну употребу.

Имплементирани алгоритми се могу користити за  препознавање облика, детекцију и препознавање лицa, праћење покрета при видео снимку, спајање више слика у једну, препознавање маркера за проширену стварност и слично. Такви алгоритми су примењени у бројним програмима попут програма за видео надзор, навигацију и аутоматизацију рада робота, проверу призвода у фабрикама, асистенцију при вожњи аутомобила и тд.

Рачунарски вид помаже у прикупљању релевантних информација са слика и доношењу одлука базираним на тим подацима. Циљ рачунарског вида је да омогући да рачунар посматра ствари на исти начин као и људи. Основни кораци система базираног на рачунарском виду су:

\begin{itemize} \itemsep1pt \parskip0pt \parsep0pt
  \item прикуљање слика
  \item манипулација сликама
  \item извлачење релевантних информација
  \item доношење одлука
\end{itemize}

NAPOMENA: Ukljuci sliku cv.png \\

Као што се види из наведеног, за један такав систем јако су важни и алгоритми машинсог учења (machine learning) као и алгоритми обраде слика (image processing). OpenCV библиотека садржи такве алгоритме.
Обрада слика је процес манипулације подацима слике у сврху пружања информација дањем току алгоритама рачунарског вида.
Компонента која нас занима је управо обрада слика. 

\section{Сиво скалирана слика}

Први корак за већину алгоритама обрадa слика је рачунање сиво скалиране слике на основу оригиналне слике, слике која садржи црвену, зелену и плаву компоненту.. У даљем процесу користи се само сиво скалирана слика. На тај начин  постиже се значајна уштеда у меморији јер слика која се обрађује може бити копирана више пута, самим тим постижу се боље перформансе,  а добија се и на једноставности алгоритама.

\begin{lstlisting}[language=Java,label=lst:grayscale,caption=Рачунање сиво скалиране слике]
// Load BGR image.
Mat bgrImage = imread(path, CV_LOAD_IMAGE_COLOR);

// Convert image to grayscale.
Mat gsImage;
cvtColor(bgrImage, gsImage, CV_BGR2GRAY);
\end{lstlisting}

 Сиво скалирана слика рачуна се као:
\begin{equation}
Y \gets 0.299 \cdot R + 0.587 \cdot G + 0.114 \cdot B
\end{equation}

где су Y - сиво скалирана слика, R - црвена компонента оригиналне слике, G - зелена компонента оригиналне слике и B - плава компонента оригиналне слике. Коефицијенти представљају измерену перцепцију интензитета код трохроматских људи. Конкретно, људски вид је најосетљивији на зелену, а најмање на плаву боју. \\

NAPOMENA: Ukljuci sliku original-gray.png

\section{Сегментација слике}

Сегментација слике је процес раздвајања објеката од позадине слике.

\section{Детекција контура}

\section{Сегментација спојених трагова}

\subsection{Ерозија}

\subsection{Слив}

\section{Одређивање угла слике}

\chapter{Развој корисничког интерфејса употребом Qt фрејмворка}

\chapter{Софтвер за детекцију трагова неутронске дозиметрије}

\chapter{Тестирање}

\chapter{Закључак}


%
% Spisak Literature
%
\begin{thebibliography}{11}
\bibitem{larsv} {Lars Vogel, Android Service and Broadcast Receiver, www.vogella.de, 2011.}
\bibitem{literatura1} ...

\bibitem{opencv} http:\slash \slash opencv.org, OpenCV званична веб страна

\end{thebibliography}

	
\end{document}


